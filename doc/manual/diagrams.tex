 
\chapter{Diagram generation}
\label{diagrams}

Starting with version 5.0, \FORM{} is equipped with the diagram generator of 
Toshiaki Kaneko: ``A Feynman graph generator for any order of coupling 
constants'', Toshiaki Kaneko (Meiji Gakuin U.), Comput.Phys.Commun. 92 (1995) 
127-152, e-Print: hep-th/9408107. Recently, he has reprogrammed it as a C++ 
library, accompanied with a manual.
\FORM{} makes use of this library and has built its own syntax around it, 
based on many years of experience working with the QGRAF generator 
(``Automatic Feynman graph generation'', Paulo Nogueira: J.Comput.Phys. 105 
(1993) 279-289) and the problems encountered with the large number of 
diagrams used with the Mincer and Forcer programs. The library has been
programmed in such a way that more features should not be too hard to
implement if required.

In version 4.3 the diagram generator was not yet complete. Effectively it 
only accepted one type of scalar particle, but one could define vertices
with any number of particles. Hence it could be used as a fast topology 
generator. The old ``topologies\_'' function has been removed in version 5.0.

For the \FORM{} implementation in version 5.0 and later, the concept of
``models'' has been introduced, as well as two new types of variable,
a number of new functions and a number of new preprocessor variables.

To begin, one must define a ``Model''\index{model}\index{diagrams!model} which contains the fields
and vertices to be used for the generation of Feynman graphs.
A Model definition is finished with ``EndModel''\index{endmodel}\index{diagrams!endmodel}.

A ``Particle''\index{particle}\index{diagrams!particle} is defined with the syntax:
\begin{verbatim}
   Particle particlename[,antiparticlename] [,<sign><spin>] [,external];
\end{verbatim}
A Particle may optionally have an antiparticle\index{antiparticle}\index{diagrams!antiparticle} name assigned to
it; if no antiparticle is specified the particle is its own antiparticle.
The \texttt{<sign>} denotes bosonic (\texttt{+}) or fermionic (\texttt{-})
statistics and the \texttt{<spin>} denotes the dimension of the particle's
spin representation; for example, a scalar is defined with \texttt{+1}, an
electron with \texttt{-2} and a gluon with \texttt{+3}. The spin information
is used by the generator to forbid invalid vertices, but does not ultimately
affect the generation of the graphs. Neglecting the sign and spin information
is equivalent to specifying \texttt{+1}. The \texttt{external} option specifies
that the Particle should appear only as an external leg.
Note that attempting to define an external scalar particle as
\texttt{Particle MyScalar,external;} will produce a Particle called
``MyScalar'' with antiparticle ``external''. This case must be defined using
\texttt{Particle MyScalar,1,external;}.

Following the Particle definitions inside the Model scope, one defines
interaction vertices. An N-point Vertex\index{vertex}\index{diagrams!vertex} is defined as follows, where
$N\geq2$:
\begin{verbatim}
   Vertex particle1,...,particleN: coupling;
\end{verbatim}
The coupling should be a (product of) symbol(s) to integer powers.
Symbols used here will be declared automatically if not already declared.

Multiple models, with different names, may be defined within a single \FORM{}
script. Some concrete examples follow:

\begin{verbatim}
   Model PHI3;
      Particle phi, +1;
      Vertex phi,phi,phi: g;
   EndModel;
\end{verbatim}

\begin{verbatim}
   Model PHI4;
      Particle phi, +1;
      Vertex phi,phi,phi,phi: g^2;
   EndModel;
\end{verbatim}

\begin{verbatim}
   Model QCD;
      Particle qua,QUA,  -2;
      Particle gho,GHO,  -1;
      Particle glu,      +3;
      Vertex QUA,qua,glu: g;
      Vertex GHO,gho,glu: g;
      Vertex glu,glu,glu: g;
      Vertex glu,glu,glu,glu: g^2;
   EndModel;
\end{verbatim}

% TODO: reinstate this example once we have it working in the new grcc setup.
%In some cases one might want to generate diagrams in which the internal 
%(multi)loop propagators are represented as special edges. This requires the 
%definition of more complicated models as in
%\begin{verbatim}
%    #do i = 0,`LOOPS'
%    Model QCD`i';
%        Particle qua,QUA,-2;
%        Particle gho,GHO,-1;
%        Particle glu,+3;
%        Vertex qua,QUA,glu:g;
%        Vertex gho,GHO,glu:g;
%        Vertex glu,glu,glu:g;
%        Vertex glu,glu,glu,glu:g^2;
%        #do j = 1,`i'-1
%            Vertex qua,QUA:g^{2*`j'};
%            Vertex gho,GHO:g^{2*`j'};
%            Vertex glu,glu:g^{2*`j'};
%        #enddo
%    EndModel;
%    #enddo
%\end{verbatim}
%in which the two-point vertices are the propagators with j loops. As one 
%can see in this example, one can have vertices with identical particle 
%contents but different powers of the coupling constants.

Once a Model has been defined, Feynman graphs may be generated with the
new ``diagrams\_''\index{diagrams\_}\index{function!diagrams\_}\index{diagrams!diagrams\_} function,
which has the following syntax:
\begin{verbatim}
   diagrams_(model_name, incoming_particle_set, outgoing_particle_set,
             external_momenta_set, internal_momenta_set,
             number_of_loops_or_couplings, options);
\end{verbatim}
Here, \texttt{model\_name} is the name of the user-defined Model,
and \texttt{incoming\_particle\_set} and \texttt{outgoing\_particle\_set} are
\FORM{} sets of Particle names, for example ``\texttt{\{phi\}}'' or
``\texttt{\{glu,glu\}}'', representing the desired scattering amplitude.
At two or more loops, it is also possible to generate vacuum graphs by
specifying empty sets for both the incoming and outgoing particles.
\texttt{external\_momenta\_set} and \texttt{internal\_momenta\_set} are \FORM{}
sets of vectors to be used to represent internal and external particle
momenta, respectively. These can be defined sets, such as
\begin{verbatim}
   Vector q1,...,q10, k1,...,k10;
   Set ext : q1,...,q10;
   Set int : k1,...,k10;
\end{verbatim}
or defined dynamically in the call of diagrams\_ by providing, for example,
\{q1,q2\}.
The vectors in the external and internal sets must not come with a minus sign,
and the sets must not contain any repeated entries.
The \texttt{number\_of\_loops\_or\_couplings} can be set either to a number
denoting the number of loops required, or to specific powers of the coupling
constants used in the Model vertices, for example
``\texttt{g\^{}2}'' or ``\texttt{gs\^{}2 * gw\^{}2}''.

Finally, the \texttt{options} argument can be used to control the generation
and output formatting of the graphs.
The options are \FORM{} pre-processor variables, and should be added together.
If no filtering options are specified
(by passing ``\texttt{0}'' or omitting the argument entirely), all connected
graphs are generated.
The graph-filtering keywords are defined to be compatible with their
counterparts in QGRAF and
are given below. Options which are inverse to each other may not be specified
simultaneously.
The user must be careful not to accidentally specify a keyword twice; this
will not have the intended effect, but rather
generate a different keyword entirely. This subtlety may be improved in the
future.
\begin{description}
    \item[\textbf{\texttt{`OnePI\_'}, \texttt{`OnePR\_'}}:]
    \index{diagrams!onepi}\index{diagrams!onepr}
    generate only one-particle irreducible (reducible) graphs.
    \item[\textbf{\texttt{`OnShell\_'}, \texttt{`OffShell\_'}:}]
    \index{diagrams!onshell}\index{diagrams!offshell}
    generate only graphs without (with) self-energy corrections on external lines.
    \item[\textbf{\texttt{`NoSigma\_'}, \texttt{`Sigma\_'}:}]
    \index{diagrams!nosigma}\index{diagrams!sigma}
    generate only graphs without (with) any self-energy corrections on any line.
    \item[\textbf{\texttt{`NoSnail\_'}, \texttt{`Snail\_'}:}]
    \index{diagrams!nosnail}\index{diagrams!snail}
    generate only graphs without (with) snails.
    \item[\textbf{\texttt{`NoTadpole\_'}, \texttt{`Tadpole\_'}:}]
    \index{diagrams!notadpole}\index{diagrams!tadpole}
    generate only graphs without (with) tadpoles.
    \item[\textbf{\texttt{`Simple\_'}, \texttt{`NotSimple\_'}:}]
    \index{diagrams!simple}\index{diagrams!notsimple}
    generate only graphs without (with) any vertices connected by two or more edges.
    \item[\textbf{\texttt{`Bipart\_'}, \texttt{`NonBipart\_'}:}]
    \index{diagrams!bipart}\index{diagrams!nonbipart}
    generate only bipartite (non-bipartite) graphs.
    \item[\textbf{\texttt{`CyclI\_'}, \texttt{`CyclR\_'}:}]
    \index{diagrams!cycli}\index{diagrams!cyclr}
    generate only cycle irreducible (reducible) graphs.
    \item[\textbf{\texttt{`Floop\_'}, \texttt{`NotFloop\_'}:}]
    \index{diagrams!floop}\index{diagrams!notfloop}
    generate only graphs which do not (do) contain closed fermion loops with an odd number of vertices.
\end{description}
Graph generation is further controlled with the options:
\begin{description}
    \item[\textbf{\texttt{`WithSymmetrizeI\_'}, \texttt{`WithSymmetrizeF\_'}:}]
    \index{diagrams!withsymmetrizei}\index{diagrams!withsymmetrizef}
    symmetrize between the initial (final) state particles. For example, when generating a boson
    propagator one could define both external particles as incoming and provide the
    \texttt{`WithSymmetrizeI\_'} option.
    \item[\textbf{\texttt{`TopologiesOnly\_'}:}]
    \index{diagrams!topologiesonly}
    generate only the distinct topologies which appear
    in the requested amplitude. In this mode, Particle information is not given in the
    output, but only the momenta flowing into each vertex.
    The topology numbering in the \texttt{topo\_}\index{topo\_}\index{function!topo\_}\index{diagrams!topo\_} tags is consistent with and without
    this option, such that graphs in the full output are produced with their topologies
    already identified. Preparatory work can be performed efficiently at the level of the 
    topologies before processing the full graph output.
\end{description}
In addition to the filtering keywords, the following options control the formatting of the output:
\begin{description}
    \item[\textbf{\texttt{`WithEdges\_'}:}]
    \index{diagrams!withedges}
    produce also \texttt{edge\_} functions which contain propagator momenta and the numbers of the vertices to which they connect.
    \item[\textbf{\texttt{`WithoutNodes\_'}:}]
    \index{diagrams!withoutnodes}
    omit the \texttt{node\_} functions, which describe the fields and momenta which flow into each vertex.
    \item[\textbf{\texttt{`WithBlocks\_'}:}]
    \index{diagrams!withblocks}
    \index{block\_}\index{function!block\_}\index{diagrams!block\_}tag each graph's ``blocks'', sub-graphs which are connected to the rest of the graph by a single vertex, in the \texttt{block\_} function.
    \item[\textbf{\texttt{`WithOnePISets\_'}:}]
    \index{diagrams!withonepisets}
    \index{onepi\_}\index{function!onepi\_}\index{diagrams!onepi\_}tag each graph's one-particle irreducible subsets of vertices in the \texttt{onepi\_} function.
\end{description}

With the above options and example models in mind, we now display some example output from the
generator. The options:
\begin{verbatim}
   Local gluglu1 = diagrams_(QCD, {glu}, {glu}, int, ext, 1,
       `WithEdges_'+`OnShell_'+`NoTadpole_'+`NoSnail_');
\end{verbatim}
produce:
\begin{verbatim}
   gluglu1 =
       -
         topo_(1)
         *node_(1,1,glu(-k1))
         *node_(2,1,glu(-k2))
         *node_(3,g,QUA(-q1),qua(-q2),glu(k1))
         *node_(4,g,QUA(q2),qua(q1),glu(k2))
         *edge_(1,glu(k1),1,3)
         *edge_(2,glu(k2),2,4)
         *edge_(3,qua(q1),3,4)
         *edge_(4,QUA(q2),3,4)
       -
         topo_(1)
         *node_(1,1,glu(-k1))
         *node_(2,1,glu(-k2))
         *node_(3,g,GHO(-q1),gho(-q2),glu(k1))
         *node_(4,g,GHO(q2),gho(q1),glu(k2))
         *edge_(1,glu(k1),1,3)
         *edge_(2,glu(k2),2,4)
         *edge_(3,gho(q1),3,4)
         *edge_(4,GHO(q2),3,4)
       +
         1/2
         *topo_(1)
         *node_(1,1,glu(-k1))
         *node_(2,1,glu(-k2))
         *node_(3,g,glu(k1),glu(-q1),glu(-q2))
         *node_(4,g,glu(k2),glu(q1),glu(q2))
         *edge_(1,glu(k1),1,3)
         *edge_(2,glu(k2),2,4)
         *edge_(3,glu(q1),3,4)
         *edge_(4,glu(q2),3,4)
      ;
\end{verbatim}

The \texttt{node\_}\index{node\_}\index{function!node\_}\index{diagrams!node\_} function arguments give their id number, the coupling
associated with the vertex they represent, and the Particle fields
which connect to them (which are functions of the incoming momenta).
The \texttt{edge\_}\index{edge\_}\index{function!edge\_}\index{diagrams!edge\_} function arguments give their id number, the
Particle and momentum of the associated propagator, and the id
numbers of the \texttt{node\_} functions which they connect.
External particles have a special \texttt{node\_} function containing a
single field and a coupling of 1.

Specifying additionally \texttt{`TopologiesOnly\_'} produces just the single
contributing topology:
\begin{verbatim}
   gluglu1 =
       +
         topo_(1)
         *node_(1,1,-k1)
         *node_(2,1,-k2)
         *node_(3,g,k1,-q1,-q2)
         *node_(4,g,k2,q1,q2)
         *edge_(1,k1,1,3)
         *edge_(2,k2,2,4)
         *edge_(3,q1,3,4)
         *edge_(4,q2,3,4)
      ;
\end{verbatim}
In this case, Particle information is omitted from the \texttt{node\_} and \texttt{edge\_} functions.
The \texttt{topo\_} tags are consistent with the tags present in the full graph
output, produced when \texttt{`TopologiesOnly\_'} is not specified.

To assist with debugging configuration problems or to see more information on
the internals of the diagram generator, one may specify ``On GrccVerbose;''.

